%TCIDATA{LaTeXparent=0,0,relatorio.tex}

%\resumo{Resumo}{O presente texto apresenta normas a serem seguidas por alunos do Curso de Engenharia Mecatr�nica da Universidade de Bras�lia para reda��o de relat�rio na disciplina Projeto de Gradua��o 2. Tais normas foram aprovadas pela Comiss�o de Gradua��o do Curso de Engenharia Mecatr�nica em julho/2005. S�o apresentadas instru��es detalhadas para a formata��o do trabalho em termos de suas partes principais.}
\resumo{Resumo}{Continuar o desenvolvimento de um sensor de geometria da po�a de fus�o em um processo de soldagem GMAW por curto circuito por meio de algoritmos de processamento de imagem que possibilitem para cada quadro:


\begin{itemize}
\item
Selecionar quadros prop�cios ao processamento;
\item
Determinar a largura e o eixo central do arame, bem como a distor��o de perspectiva que ocorre no seu plano normal � dire��o do movimento de soldagem;
\item
Determinar a largura m�xima da po�a, assim como a distor��o de perspectiva que ocorre no plano da po�a de soldagem;
\end{itemize}

Os resultados do processamento acima ser�o organizados em vetores, os quais dever�o ser filtrados por um Filtro de Kalman para gera��o de sinais prop�cios � utiliza��o em uma malha de controle. Os algoritmos dever�o ser validados comparando seus resultados com valores medidos � partir de testes controlados do processo.
}

\vspace*{2cm}

\resumo{Abstract}
{
Continue the development of a welding pool geometry sensor in a short circuit GMAW process by means of the development of an image processing algorithm that makes possible at each frame:

\begin{itemize}
\item
Select frames appropriate to processing;
\item
Determine the wire's width and central axis, as well as perspective distortion in its plane normal to the welding's movement direction;
\item
Determine the pool's max width, as well as the distortion perspective in the pool's plane;
\end{itemize}

The results of the processing above will be organized in vectors, which should be filtered by a Kalman Filter for the generation of  signals appropriate for utilization in a control system. The algorithms should be validated comparing its results with measured values from controlled process tests.
}