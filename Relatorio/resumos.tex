%TCIDATA{LaTeXparent=0,0,relatorio.tex}

%\resumo{Resumo}{O presente texto apresenta normas a serem seguidas por alunos do Curso de Engenharia Mecatr�nica da Universidade de Bras�lia para reda��o de relat�rio na disciplina Projeto de Gradua��o 2. Tais normas foram aprovadas pela Comiss�o de Gradua��o do Curso de Engenharia Mecatr�nica em julho/2005. S�o apresentadas instru��es detalhadas para a formata��o do trabalho em termos de suas partes principais.}
\resumo{Resumo}{
Desenvolvimento de m�todos para a obten��o em tempo real de par�metros em um processo de soldagem GMAW (\textit{Gas Metal Arc Welding}) por meio de processamento de imagens. Em uma primeira etapa foi desenvolvido um algoritmo em software que realiza as seguintes tarefas a cada quadro: 

\begin{itemize}
\item
Selecionar quadros prop�cios ao processamento;
\item
Determinar a largura e o eixo central do arame, bem como a distor��o de perspectiva que ocorre no seu plano normal � dire��o do movimento de soldagem;
\item
Determinar a largura m�xima da po�a, assim como sua posi��o.
\end{itemize}

Na segunda etapa, algumas das tarefas acima foram implementadas em \textit{hardware} com o uso de um FPGA (\textit{Field Programmable Gate Array}) para reduzir significativamente o tempo de processamento.
}

\vspace*{2cm}

\resumo{Abstract}
{
%Continue the development of a welding pool geometry sensor in a short circuit GMAW process by means of the development of an image processing algorithm that makes possible at each frame:
The development of image processing methods for real time calculation of parameters in a GMAW (Gas Metal Arc Welding). A software algorithm was created in the first stage to realize the following task at each frame:

\begin{itemize}
\item
Select proper frames for processing;
\item
Determine the wire's width and central axis, as well as perspective distortion in the plane orthogonal to the welding's movement direction;
\item
Determine the pool's max width and position.
\end{itemize}

In the second stage, some of the tasks above were implemented in hardware with the use of a FPGA (Field Programmable Gate Array) to reduce significantly the processing time.

}