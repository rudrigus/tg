%TCIDATA{LaTeXparent=0,0,relatorio.tex}
                      

\chapter{Conclus�es}\label{CapConclusoes}

Foi interessante observar o efeito que cada filtro tem sobre o resultado final. Mais importante que isso foi encontrar um conjunto de filtros, que aplicados na ordem certa, transformam a imagem em algo mais f�cil de trabalhar.

A implementa��o em FPGA foi mais dif�cil e demorada do que se esperava. Foram encontrados diversos problemas durante o desenvolvimento que n�o tinham solu��o trivial. 

Sugere-se que sejam criados os blocos finais na implementa��o em FPGA, o de regress�o robusta e o de corre��o de distor��o de perspectiva. Esses blocos tornariam a solu��o completamente funcional para uma aplica��o em ind�strias ou \textit{workshops}.

Infelizmente n�o foi poss�vel utilizar a c�mera para testes por falta do \textit{driver} de comunica��o, portanto sugere-se que esta etapa seja feita, sem descartar a possibilidade de se utilizar outro modelo de c�mera.

Os mesmos m�todos possivelmente podem ser aplicados para se encontrar a abertura entre duas placas met�licas em um processo de soldagem. Com essa informa��o poderia-se ter um conhecimento bastante �til da dist�ncia entre a ponta do eletrodo de solda e o centro da abertura entre as placas.

-----------------------------------------------------------
Este cap�tulo � em geral formado por: um breve resumo do que foi apresentado, conclus�es mais pertinentes e propostas de trabalhos futuros.
